\chapter{Conclusion}
Este trabalho tem como objetivo propor e desenvolver uma técnica de \textit{Macro Placement} com posicionamento de \textit{pads} de forma iterativa. Este desenvolvimento ocorrerá dentro de uma plataforma \textit{open source}, para que possa servir como base para outras propostas e abordagens sobre \textit{Macro Placement}.

Através de um estudo teórico, foi apresentado o processo de síntese física, e especificado sobre as necessidades e dificuldades da etapa de \textit{Chip Planning}, com foco no processo de \textit{Macro Placement}. Uma revisão bibliográfica permitiu identificar trabalhos correlatos que apresentam abordagens diferentes para o problema de \textit{Macro Placement}, tanto com novas abordagens como Inteligencia Artificial, como através de revisitando técnicas tradicionais e adicionando novos processos, mostrando que ainda há muito espaço para diferentes abordagens.

Com estas considerações, este trabalho propõe uma técnica que utiliza da mesclagem de etapas normalmente bem definidas e separadas, o posicionamento de \textit{pads} e 
 \textit{Macro Placement}, para melhor realizar um \textit{Chip Planning}.

 Foram realizados experimentos iniciais utilizando alterações no fluxo de síntese padrão da plataforma \textit{OpenROAD} para validar a proposta e explorar qual abordagem ela deve possuir. Ainda há muito desenvolvimento e testes a serem feitos, que devem ser realizado no Trabalho de Conclusão de Curso 2.