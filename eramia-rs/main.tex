\documentclass[12pt]{article}

\usepackage{sbc-template}
\usepackage{graphicx,url}
\usepackage[utf8]{inputenc}
\usepackage[brazil]{babel}  

     
\sloppy

\title{Detecção de Documentos Acadêmicos Falsificados: Uma Solução Baseada em Aprendizado de Máquina}

\author{Samuel M. Ransolin\inst{1}, Giovana N. Inocêncio\inst{1}, Jean E. Martina\inst{1} }

\address{Departamento de Informática e Estatística -- Centro Tecnológico \\
  Universidade Federal do Santa Catarina (UFSC) -- Florianópolis, SC -- Brasil
\email{samuel.moreira.ransolin@grad.ufsc.br, \{giovana.inocencio,jean.martina\}@ufsc.br}
}

\begin{document} 

\maketitle

\begin{abstract}
  FIXME
\end{abstract}
     
\begin{resumo} 
  Nos últimos anos, no Brasil, o crescimento de ingressantes, de formandos e de instituições de ensino superior intensificou os desafios relacionados à validação de certificados acadêmicos, já que a verificação é majoritariamente manual, sujeita a erros e a aceitação de fraudes.
  Este trabalho revisita o estado-da-arte em detecção de documentos falsificados via aprendizado de máquina, e propõe um protótipo híbrido que combina análise multimodal, clustering, detecção de anomalias e classificação por grau de legitimidade.
  Ao integrar o protótipo à Jornada do Estudante, documentos podem ser validados automaticamente antes do registro em sua rede distribuída, aumentando a segurança e a confiabilidade do credenciamento.
  % Nesse contexto, a Jornada do Estudante (MEC) oferece acompanhamento de registros acadêmicos por meio de uma rede distribuída permissionada, permitindo que apenas instituições comprovadas registrem créditos e certificações.
\end{resumo}


\section{Problemática}

Com o decorrer da última década, presencia-se no Brasil crescimento contínuo na emissão de diplomas de ensino superior, com cifras que chegam ao aumento de mais de 31\% de formandos desde 2013 \cite{mec:2023}. Embora isso revele um saldo extremamente positivo, também traz à tona desafios que precisam ser superados, entre eles e temática explorada neste trabalho, a melhoria nos processos de regulação, supervisão e avaliação dessas emissões por parte do Ministério da Educação do Brasil (MEC).

Atualmente, a gerência, armazenamento e emissão de documentos acadêmicos, como diplomas e históricos escolares, é responsabilidade da instituição de ensino que os emite \cite{mec:1978}. Além disso, o processo, burocrático e não computadorizado, é suscetível a erros e até mesmo fraudes devido à ausência de transparência e redundância \cite{palma:2019}. Assim, essa falta de modernização deixa brechas que são conhecidas e utilizadas por agentes mal-intencionados, possibilitando a criação de falsas instituições, especializadas na venda de pacotes que incluem diversos certificados contrafeitos amparados em documentos oficiais adulterados, de forma a conferir aparência de legalidade a diplomas sem qualquer base acadêmica real \cite{dias:2022}.

É neste cenário que o MEC, em parceria com o Ministério da Economia e diversas universidades federais, disponibiliza o sistema da Jornada do Estudante, que permite que discentes acompanhem suas trajetórias estudantis junto ao acesso a seus documentos acadêmicos pertinentes. Além disso, esse sistema também pode tornar-se uma plataforma conjunta para a emissão e registro destes certificados e até mesmo dados regulatórios das instituições de ensino superior \cite{rnp:2023}. Em acordo a essa iniciativa, o presente trabalho trata da implementação e validação de um protótipo de software que combina diferentes técnicas de aprendizado de máquina, capaz de identificar certificados falsos antes de sua inserção neste ambiente.

\section{Estado da Arte} \label{sec:firstpage}

A pesquisa acadêmica sobre identificação de documentos falsificados é escassa, especialmente quando comparada aos estudos sobre detecção de fraudes. Enquanto a detecção de fraudes foca em adulterações de arquivos originais (como mudança de notas, datas ou nomes), a identificação de documentos falsificados busca reconhecer aqueles completamente forjados desde sua criação, sem emissão por instituições oficiais. A área é predominantemente abordada através de técnicas de visão computacional, incluindo autoencoders convolucionais para análise de tintas em imagens hiperespectrais, métodos não supervisionados baseados em correlações espectrais, e reformulação do problema como comparação de grafos via OCR.

As abordagens mais robustas combinam múltiplas tecnologias para melhorar a detecção. O trabalho de integração entre blockchain e aprendizado de máquina para diplomas propõe um pipeline de quatro etapas: pré-processamento em nuvem, detecção de artefatos suspeitos com Faster R-CNN, segmentação pixel-level com Mask R-CNN, e consenso distribuído através de algoritmos de ML multicamadas. Paralelamente, pesquisas em análise multimodal demonstram a eficácia da fusão de características textuais e visuais, combinando extração semântica via ULMFiT e FastText com features visuais de redes convolucionais, alcançando 93,6\% de acurácia na classificação de documentos.

Abordagens alternativas tratam a autenticação como problema de agrupamento, utilizando K-means para identificar padrões consistentes em documentos legítimos e classificando novos documentos pela proximidade aos centroides estabelecidos, atingindo 86,53\% de acurácia.

\section{Metodologia}


\bibliographystyle{sbc}
\bibliography{sbc-template}

\end{document}
